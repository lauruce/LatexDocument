% 导言区
\documentclass[a4paper, 11pt]{article} 

% 导入中文宏
\usepackage{ctex}

\title{\heiti Linux操作指南}
\author{\kaishu 黄成都}
\date{\today}
% 正文区
\begin{document}
    \maketitle
    \thispagestyle{empty}
    
    \newpage
    \tableofcontents
    \thispagestyle{empty}

    \newpage
    \setcounter{page}{1}
    \section{磁盘操作}
    \subsection{分区}
        一块新的磁盘,加入系统后,使用fdisk -l命令可以查看当前加入的磁盘。fdisk /dev/sda
        进入磁盘操作命令界面,输入'g'创建一个GPT分区表,然后就可以使用'n'来创建新分区。分区
        创建好后使用'w'写入并保存。

        mkfs.ext4 /dev/sda1来格式化sda1分区,并创建ext4文件系统,完成后sda1分区就可以挂载
        到系统的任何地方了。

        df -h则可以查看当前磁盘的挂载和使用情况。

    %\newpage
    %\section{第二章}     
    %\subsection{第二点一章}

    %\begin{center}    
    %\end{center} 

\end{document}